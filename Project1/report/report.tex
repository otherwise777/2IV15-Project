\documentclass[a4paper,twoside,11pt,twocolumn]{article}
\usepackage{a4wide,graphicx,fancyhdr,amsmath,amssymb,float,longtable, chronology}
\usepackage{algorithmic}
\usepackage{hyperref}
\usepackage{url}

%----------------------- Macros and Definitions --------------------------

\setlength\headheight{20pt}
\addtolength\topmargin{-10pt}
\addtolength\footskip{20pt}

\newcommand{\N}{\mathbb{N}}
\newcommand{\ch}{\mathcal{CH}}
\everymath{\displaystyle}
\newcommand{\solution}[1]{\noindent{\bf Solution to Exercise #1:}}
\newcommand{\scg}{Simulation in Computer Graphics}

\fancypagestyle{plain}{%
\fancyhf{}
\fancyhead[LO,RE]{\sffamily\bfseries\large Technische Universiteit Eindhoven}
\fancyhead[RO,LE]{\sffamily\bfseries\large 2IV15 \scg}
\fancyfoot[LO,RE]{\sffamily\bfseries\large Department of Mathematics and Computer Science}
\fancyfoot[RO,LE]{\sffamily\bfseries\thepage}
\renewcommand{\headrulewidth}{0pt}
\renewcommand{\footrulewidth}{0pt}
}

\pagestyle{fancy}
\fancyhf{}
\fancyhead[RO,LE]{\sffamily\bfseries\large Technische Universiteit Eindhoven}
\fancyhead[LO,RE]{\sffamily\bfseries\large 2IV15 Simulation in Computer Graphics}
\fancyfoot[LO,RE]{\sffamily\bfseries\large department of mathematics and computer science}
\fancyfoot[RO,LE]{\sffamily\bfseries\thepage}
\renewcommand{\headrulewidth}{1pt}
\renewcommand{\footrulewidth}{0pt}

%-------------------------------- Title ----------------------------------

\title{\sffamily\bfseries 2IV15 \scg\ - Project 1}
\author{Arno Tiemersma \qquad Student number: 0716959 \\{\tt a.w.g.tiemersma@student.tue.nl}\\ \\Mart Pluijmaekers \qquad Student number: 0753117 \\{\tt m.h.l.pluijmaekers@student.tue.nl}}

\date{\today}

%--------------------------------- Text ----------------------------------

\begin{document}
\maketitle
%\tableofcontents
%\newpage
\section{Introduction}
This paper describes our implementation of a particle system that provides generalized force and constraint structures, a number of numerical integration schemes, basic cloth simulation and interaction with the particles in the system.

\section{Generalized force structure}
Particles are defined as having a position, a velocity, a mass and a force accumulator. Every simulation step, the current position and velocity are updated by calculating all the forces that act on a specific particle, and then solving the following differential equations:
\begin{align}
	\dot{\mathbf{x}} &= \mathbf{v}\\
	\dot{\mathbf{v}} &= \mathbf{a} = \mathbf{F}(\mathbf{x},\mathbf{v})/m
\end{align}
Every force inherits from a general \texttt{class Force} and implements the \texttt{Force::act()} method. All the forces in the system are stored in a \texttt{std::vector<Force*>}, which is iterated over during a simulation step. During such a step, the force accumulators in all the particles are set to zero, after which \texttt{Force::act()} is called for every force in the system. Once all the forces have been calculated and added to the relevant force accumulators, one of the numerical solving functions (Section \ref{Numerical_schemes}) is used to calculate the next state of the system.
\subsection{GravityForce}
Gravity is calculated by multiplying the mass of a particle with a gravitational constant. In our implementation, a \texttt{Gravity} object acts on one single particle, which means that an object is added for every particle in the scene.
\subsection{SpringForce}
A \texttt{SpringForce} is a force that is defined by a rest length $\mathbf{r}$, a spring constant $k_s$ and a damping constant $k_d$. The resulting force is a function of the relative position $\mathbf{l = p1 - p2}$ and velocity $\mathbf{\dot{l} = v1 - v2}$ of the particles and the specification of the spring, as shown in formula \ref{springforce}. 

\begin{align} \label{springforce}
	\mathbf{f_{p1}} &= \bigg[k_s(|\mathbf{l}| - \mathbf{r}) + k_d \frac{\dot{\mathbf{l}}\cdot \mathbf{l}}{|\mathbf{l}|}\bigg]\frac{\mathbf{l}}{|\mathbf{l}|}\\
	\mathbf{f_{p2}} &= -\mathbf{f_{p1}}
\end{align}

A \texttt{SpringForce} object is added for every pair of particles in the scene that is connected by a spring.

\subsection{AngularForce}
An \texttt{AngularForce} object acts like a spring force on a triplet of particles $(\textbf{p1, p2, p3})$, pulling the particles so that the angle between $\textbf{p1} - \textbf{p2}$ and $\textbf{p2} - \textbf{p3}$ approaches some rest angle $\alpha_{rest}$. This can be used to build a basic hair-simulation.

\section{Generalized constraint structure}

\section{Numerical integration schemes}
\label{Numerical_schemes}

\section{Interaction}

\end{document}
